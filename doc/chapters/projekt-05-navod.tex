\documentclass[../projekt.tex]{subfiles}
\begin{document}


\chapter{Návod}

Nástroj isamon je program spustitelný z příkazové řádky používaný ke skenování aktivních hostů v IPv4 sítích a jejich portů.


\begin{lstlisting}[caption=Možnosti použití nástroje isamon.]
isamon [-h] [-i <interface>] [-t] [-u] [-p <port>] [-w <ms>] -n <net_address/mask>
\end{lstlisting}

\section{Parametry}

Povinné parametry:
\begin{itemize}
    \item \textbf{-n} <adresa\_sítě/maska> - adresa a mask sítě, která má být skenována ve formátu CIDR\cite{RFC4632}.  
\end{itemize}

Nepovinné parametry:
\begin{itemize}
    \item \textbf{-h} - vypíše nápovědu k užívání programu isamon
    \item \textbf{-i} <rozhrání> - specifikuje rozhraní, které má být použito ke komunikaci
    \item \textbf{-t} - určuje, že má dojít ke skenování otevřených TCP portů
    \item \textbf{-u} - určuje, že má dojít ke skenování otevřených UDP portů
    \item \textbf{-p} <port> - zobrazí aktivní klienty, kteří mají otevřený tento port
    \item \textbf{-w} <ms> - maximální povolený RTT v ms
\end{itemize}


\section{Ukázky použití}

\begin{lstlisting}[caption=Ukázka použití isamon pro skenování TCP a UDP portů sítě 192.168.1.0/30 s maximálním RTT 5 ms.]
$ isamon -n 192.168.1.0/30 -t -u -w 5 
192.168.1.1
192.168.1.1 TCP 80
192.168.1.1 TCP 22
192.168.1.1 UDP 53
\end{lstlisting}

\begin{lstlisting}[caption=Ukázka použití isamon pro skenování aktivních hostů na síti 192.168.1.0/30.]
$ isamon -n 192.168.1.0/30
192.168.1.1
192.168.1.2
\end{lstlisting}

\end{document}